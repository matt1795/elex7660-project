\section*{Abstract}

The main objective of this project is to generate a black and white composite
video signal that will drive a Cathode Ray Tube (CRT) television. To accomplish
this, we will use hardware to create a basic video card for a CRT TV. We will be
able to achieve a resolution of 240 x 320 pixels on the screen.

The major components of this design include a digital interface for a processor,
we will first use SPI, as it is simple to implement and will make the use of
this device comparable to the Nokia 5110 LCD screen. Secondly, a storage buffer
will need to be implemented that can handle different rates of data being
written to and read from it. Lastly, there will be an encoder module which
parses the storage buffer and generates the composite video via DAC.

We currently have a black and white CRT television that will allow us to
accomplish most of our goals. We will need to find a colour television if we get
to the point of adding colour capability to the video card. For driving
electronics, we also already have what we need in order to implement a simple,
lopsided resistor DAC that will generate the correct voltage levels for
composite video. Once we start adding shading, we will have to incorporate a
higher quality DAC. At first glance, we need this DAC to have a parallel input
to achieve fast conversion rates, and either 8 or 12 bit precision. 

Out project goals are in order as follows. Controlling the CRT TV to display an
alternating black and white signal. Increasing the depth of the black and white
inputs to be able to display shades. Connecting a MSP430 through a SPI input and
displaying the preprogrammed pong game on the CRT TV. Introducing color to our
black and white CRT TV by using phase modulation superimposed on the analog
input. Finally, we would add sound to our composite video card.
