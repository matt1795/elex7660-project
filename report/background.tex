\section{Background}

\subsection{The Composite Video Signal}

Originally video was broadcast in black and white, only requiring one channel.
In terms of driving the television itself, one needs only change DC voltage in
the active video region of the signal to change luminosity levels.

Once colour televisions became a reality, video needed to be broadcast in colour
as well. In order to also be backwards compatible with black and white
televisions still in common use, colour was added to the video signal by adding
an amplitude and phase modulated sinusoid. This allowed for only requiring one
additional radio band for broadcast, keeping the now "composite" signal on a
single wire, and black and white televisions would discard the colour components
by passing the signal through a low-pass filter.

\subsubsection{}

\subsection{}
