\section{Background}

\subsection{The Composite Video Signal}

Originally video was broadcast in black and white, only requiring one channel.
In terms of driving the television itself, one needs only change DC voltage in
the active video region of the signal to change luminosity levels.

Once colour televisions became a reality, video needed to be broadcast in colour
as well. In order to also be backwards compatible with black and white
televisions still in common use, colour was added to the video signal by adding
an amplitude and phase modulated sinusoid. This allowed for only requiring one
additional radio band for broadcast, keeping the now "composite" signal on a
single wire, and black and white televisions would discard the colour components
by passing the signal through a low-pass filter.

In the CRT television is an electron beam. In order to generate images, this
beam sweeps from left to right and changes intensity. The brightness of the
point at which the beam is directed to is proportional to the intensity of the
electron beam. One sweep of the electron beam creates a single line, and it will
do this several hundred times in order to generate a single frame on the
television. In North America, televisions will have frame rates of 30 Hz
\cite{videoBasics}, \cite{analogVideo}, \cite{video101}, \cite{rastor}.

\subsubsection{Input Output Machine}

The simplest way to view this project is as an input output machine. The input
being any device that could drive a 9-bit input and a clock. This input was 
basically, an image broken down into pixels and colours. Our machine would 
store the input signal and covert it to a composite video output. The composite
video output is complicated and requires precise timing, and precise repetition 
at a high frequency. The purpose of this was to be able to take any device as 
an input we could convert it to a component video output.

