\section{Implementation}

\begin{figure}[H]
    \centering
    \caption{Module Diagram}
\end{figure}

\subsection{8-bit DAC Hardware}


\subsection{Composite Driver}

\subsubsection{Timing Control}

\subsubsection{Line Buffer}

\subsubsection{Colour Modulation}

\subsection{SPI Module}

The SPI module was designed to take a 9-bit input and a clock input. 8 of the
9 bits was to indicate colour. The last bit was for internal communication. 
Internal communication included setting individual pixels, resetting the frame 
and any other special features the input device could use. Our SPI module was
designed this way so any device, slow or fast, could be used as an input.

Once the SPI module obtained an input, the module would stack two input pixels
and store the input on an array 320 elements long. Each element on the array 
would store 16-bits or two pixels. This make for easy conversion to the SDRAM.
The SDRAM stores 16 bits worth of data at a time.


\subsection{RAM Interface}

The RAM interface was not completed. Although to get this working we would 
needed to use NIOS 2 software to give us a file to be able to access the 
SDRAM. Once this Verilog file would be created we would have to modify it to
fit our needs and properly implement it in our system.


