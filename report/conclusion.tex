\section{Conclusion}
This was a challenging project that was enjoyable and helped us understand HDL
in greater detail. Not only did this project require understanding and 
implementing an already existing video transmission method, it required us 
to deal with high frequency components, precise timing, multiple device 
communication, and temporary data buffers (SDRAM or Large arrays).

Although we did not reach all our initial goals, this project was a success. 
I can say this because we were able to control the image displayed on the black
and white TV. We would also be able to display this in 8-bit colour. Lastly, 
with just a day longer I would confidently say we would be able display an 
image from an input source.

If we were to change anything about our project we would have wanted to implement
the SDRAM because then we wouldn't be limited by resolution. This would also
let us easily expand our project to other video outputs.


\subsection{Future Directions}

Given more time, colour could easily be generated with this device. The colour
palette could be improved so that 256 colour could be implemented as there was
room in the colour encoding word size, Only the look up tables would need to be
regenerated. 

Also, later in the project it became clear that it would be advantageous for the
composite video driver to 
