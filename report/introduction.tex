\section{Introduction}

\subsection{Letter from Matt Knight}

At a party a couple years ago, a friend who worked in film approached me and
told me about a niche market in film that required the syncing of frame rates of
monitors to shutter rates of cameras. One of the most less controllable of
monitors were CRT televisions as their frame rate could not be changed, unlike
many digital monitors. While there were old converter devices that handled this,
these were made by "weird dudes in their basements"i. These induviduals would
now charge large sums of money for new devices as they did not want to make any
more.

This conversation got me researching composite video signals and the NTSC
standard. This research changed my goal to generating composite video digitally.
The idea was that one could use an old analog television as a simple display for
whatever project they chose. I first attempted to generate my own video source
from a microcontroller, however I easily ran out of processing power for a black
and white pixel display, let alone something with greyscale output or even
colour.

And instead of turning to a bigger, faster processor, it made sense to move
towards a hardware solution, and that is where FPGA's come in.

- Matt Knight

\subsection{Composite Video Driver for ELEX 7660}

This project is a video card that takes a 9-bit parallel interface to transfer
video data, and outputs the analog signal required for composite video. The
purpose of this device is to provide an easy to use, and efficient method to
display an image or video on a CRT television. There is little practicality in
this project since it is based on such an old technology, it was chosen more for
a challenging goal that suited the premise for using FPGA's: software is not
fast enough.

Our project is hosted on our github repository at \cite{git}.
